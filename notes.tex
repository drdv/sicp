\documentclass[12pt,a4paper]{article}

% ================================================================================

\newcommand{\norm}[1]{\left\lVert#1\right\rVert}
\newcommand{\minimize}[1]{\mathop{\mbox{minimize}}_{#1} \ \ }
\newcommand{\maximize}[1]{\mathop{\mbox{maximize}}_{#1} \ \ }
\newcommand{\st}[0]{\mbox{subject to} \ \ }
\newcommand\cc[1]{\texttt{#1}}
\newcommand{\abs}[1]{\left|#1\right|}

% ================================================================================

\usepackage{array}
\usepackage{amssymb}
\usepackage{amsfonts}
\usepackage{graphicx}
\usepackage[sans]{dsfont}
\usepackage{bbm}
\usepackage{amsmath,bm}
\usepackage{url}
\usepackage{hyperref}
\usepackage{eurosym}
\usepackage{bbold}
\usepackage[top=2cm, bottom=2cm, left=2cm, right=2cm]{geometry}

\def\UrlBreaks{\do\/\do-}  % https://tex.stackexchange.com/a/561193

\usepackage[toc,page]{appendix}

\usepackage[breakable]{tcolorbox}
\DeclareRobustCommand{\mybox}[2][gray!20]{
  \begin{tcolorbox}[
      breakable,
      left=0pt,
      right=0pt,
      top=0pt,
      bottom=0pt,
      colback=#1,
      colframe=#1,
      width=\dimexpr\textwidth\relax,
      enlarge left by=0mm,
      boxsep=5pt,
      arc=0pt,outer arc=0pt,
    ]
    #2
  \end{tcolorbox}
}

\begin{document}

\section{Exercise 1.13}

\subsection{Proof 1}

The second-order difference equation $F_k = F_{k-1} + F_{k-2}$ can be represented as two
first-order equations (where $n \geq 2$)
\begin{align*}
  \underbrace{\begin{bmatrix} F_{k-1} \\ F_k \end{bmatrix}}_{x_{k+1}} =
  \underbrace{\begin{bmatrix} 0 & 1 \\ 1 & 1 \end{bmatrix}}_{A}
  \underbrace{\begin{bmatrix} F_{k-2} \\ F_{k-1} \end{bmatrix}}_{x_k}.
\end{align*}
Hence, $x_{n}$ depends on the initial condition $x_0 = (F_0, F_1) = (0, 1)$ as follows
\begin{align} \label{eq.initial-condition}
  x_n = A^nx_0.
\end{align}
The eigenvalues of $A$ are~\footnote{We have to find $\lambda$ such that $\cc{det}(A - \lambda I) = 0$.}
\begin{align*}
  \lambda_1 = \frac{1 + \sqrt{5}}{2}, \quad \lambda_2 = \frac{1 - \sqrt{5}}{2},
\end{align*}
with corresponding eigenvectors~\footnote{Found by solving equations $(A - \lambda_i
I)q_i = 0$ (and normalizing).}
\begin{align*}
  q_1 = \frac{1}{n_1}\begin{bmatrix} 1 \\ \lambda_1 \end{bmatrix}, \quad
  q_2 = \frac{1}{n_2}\begin{bmatrix} 1 \\ \lambda_2 \end{bmatrix},
\end{align*}
where $n_k = \norm{q_k} = \sqrt{1 + \lambda_k^2}$. Using
\begin{align*}
  Q = \begin{bmatrix}q1 & q2\end{bmatrix}, \quad
    \Lambda = \begin{bmatrix} \lambda_1 & 0 \\ 0 & \lambda_2 \end{bmatrix}
\end{align*}
we can perform a similarity transform (note that $Q$ is an orthogonal matrix)
\begin{align*}
  x_{k+1} = \underbrace{Q\Lambda Q^T}_{A} x_k,
\end{align*}
Hence, we can express~\eqref{eq.initial-condition} as (note that $(Q\Lambda Q^T)(Q\Lambda Q^T) = (Q\Lambda^2 Q^T)$)
\begin{align*}
  x_{n}
  = Q\Lambda^n \underbrace{Q^T x_0}_{\tilde{x}_0}
  = \frac{\lambda_1}{n_1}q_1\lambda_1^n + \frac{\lambda_2}{n_2}q_2\lambda_2^n,
\end{align*}
where $\tilde{x}_0 = (\lambda_1/n_1, \lambda_2/n_2)$ can be seen as the initial
conditions after the similarity transform. Therefore, we can express the $n$-th
Fibonacci number as the first component of $x_{n}$
\begin{align*}
  F_n
  &= \frac{\lambda_1}{n_1^2}\lambda_1^n + \frac{\lambda_2}{n_2^2}\lambda_2^n \\
  &= \frac{1 + \sqrt{5}}{5 + \sqrt{5}}\lambda_1^n + \frac{1 - \sqrt{5}}{5 - \sqrt{5}}\lambda_2^n
  = \frac{\sqrt{5}\lambda_1^n - \sqrt{5}\lambda_2^n}{5} \\
  &= \frac{\lambda_1^n - \lambda_2^n}{\sqrt{5}}.
\end{align*}

Note that $|\lambda_2^n| < 1$ for any $n > 0$ and hence, dividing by $\sqrt{5} > 2$
leads to a number smaller than $1/2$. Therefore, $F_n$ is the closest integer to
$\lambda_1^n / \sqrt{5}$.

\subsection{Proof 2 (by induction)}

\begin{itemize}
\item for $n = 0$ we have $F_0 = \frac{\lambda_1^{0} - \lambda_2^{0}}{\sqrt{5}} = 0$
\item for $n = 1$ we have $F_1 = \frac{\lambda_1^{1} - \lambda_2^{1}}{\sqrt{5}} = 1$
\item next, assume that $F_{n-2} = \frac{\lambda_1^{n-2} - \lambda_2^{n-2}}{\sqrt{5}}$
  and $F_{n-1} = \frac{\lambda_1^{n-1} - \lambda_2^{n-1}}{\sqrt{5}}$, we have to prove
  that $F_{n} = F_{n-1} + F_{n-2} = \frac{\lambda_1^{n} - \lambda_2^{n}}{\sqrt{5}}$:
\end{itemize}
\begin{align*}
  \lambda_1^{n-2} - \lambda_2^{n-2} + \lambda_1^{n-1} - \lambda_2^{n-1}
  &= \lambda_1^{n-2} + \lambda_1^{n-1} - (\lambda_2^{n-2} + \lambda_2^{n-1}) \\
  &= \lambda_1^n\lambda_1^{-1}\underbrace{(\lambda_1^{-1} + 1)}_{\lambda_1} - \lambda_2^n\lambda_2^{-1}\underbrace{(\lambda_2^{-1} + 1)}_{\lambda_2} \\
  &= \lambda_1^n - \lambda_2^n,
\end{align*}
where we have used that $\lambda_i - \frac{1}{\lambda_i} = 1$ (for $i=1,2$). This is the
case because $\lambda_1$ and $\lambda_2$ are eigenvalues of $A$
(see~\eqref{eq.initial-condition}) and hence they satisfy its characteristic equation
$\cc{det}(A - \lambda I) = 0$ (\emph{i.e.,} $\lambda^2 - \lambda -1 = 0$).

\section{Quotes}

\begin{itemize}
\item (Lecture 1A) It is very easy to confuse the essence of what you are doing with the
  tools that you use.
\end{itemize}

\nocite{*}
\bibliographystyle{ieeetr}
\bibliography{bib}

\end{document}
